\documentclass{article}
\usepackage{geometry}
\geometry{a4paper,scale=0.8}
\usepackage[utf8]{inputenc}

\usepackage{amsmath, amssymb}
\usepackage{amsthm}
\usepackage{enumitem}
\usepackage{mathrsfs}
\usepackage{tikz-cd}

\setlist[enumerate,1]{
    label=(\alph*), 
    topsep=0pt, 
    partopsep=0pt, 
    parsep=0pt,
    leftmargin=*
}
\newcommand{\currchapter}{1}
\newtheoremstyle{breakstyle}
	{5pt}
	{10pt}
	{\normalfont}
	{0pt}
	{\bfseries}
	{}
	{\newline}
	{}
\theoremstyle{breakstyle}
\newtheorem{exercise}{Exercise}[section]
\renewcommand{\theexercise}{\currchapter.\thesection.\Alph{exercise}}
\title{Chapter 1 - Just enough category theory to be dangerous}
\author{Flandre}
\date{29 Dec 2025}

\begin{document}
\maketitle

\section{Categories and functors}
\begin{exercise}
	\leavevmode \vspace{-\baselineskip}
		\begin{enumerate}
			\item Since there's only one object, namely $e$, in the category, all the morphisms in such category will send $e$ to $e$.
				\begin{itemize}
					\item The combinations of mrophisms are always associative.
					\item There must exist an identity morphism, which'll be the identity in the group.
					\item Since it's a groupoid, every morphism has a inverse.
				\end{itemize}
				\emph{ (In fact every group can be represented using the language of vategory. Consider Cayley's Theorem and we ca see every group element can be regarded as an automorphism, satisfying the case of morphism to a object itself.) }
			\item We consider a category $\mathscr{C}$ whose objects are $A,B$. Consider
				$$
				\mathrm{Mor}(A,B)=\{id_{A},id_{B}\}
				.$$ 
				and we can see that is's a groupiod without being associative, therefore not a group.
		\end{enumerate}
\end{exercise}

\begin{exercise}
	\leavevmode \vspace{-\baselineskip}
		\begin{enumerate}
			\item Similar to the case in \textbf{1.1.A}, we can use exactly the same proof to show that invertible elements of $\mathrm{Mor}(A,A)$ forms a group.
			\item Automorphism groups of objects in \textbf{Example 1.1.2} are those permutation groups permutating the elements in a group.
			\item Automorphism groups of objects in \textbf{Example 1.1.3} are those invertible linear transformations mapping a linear space to itself.
		\end{enumerate}
\end{exercise}

\begin{exercise}
	% \leavevmode \vspace{-\baselineskip}
	% 	\begin{enumerate}
	% 		\item In the case of finite cases, the set of bases of $Vec_{k},Vec_{k}^{\vee}$ is finite. Therefore the functor $(\cdot)^{\vee}$ must be isomorphism.
	% 	\end{enumerate}
	\emph{skipped (require \textbf{1.1.21} to solve.)}
\end{exercise}

\begin{exercise}
	Since it's finite, just find out the bases and we can define a trivial isomorphism between $\mathscr{V}$ and $f.d.Vec_{k}$ by mapping those bases respectively.
\end{exercise}

\section{Universal properties determine an object up to unique isomorphism}
\begin{exercise}
	\leavevmode \vspace{-\baselineskip}
	\begin{enumerate}
		\item For any two initial objects $A,B$, we define
			 $$
			\varphi: A\mapsto B,\psi: B\mapsto A
			.$$ 
			And we can see 
			$$
			\varphi \circ \psi :B\mapsto B,\psi\circ\varphi:A\mapsto A
			.$$ 
			Also, notice since $A,B$ are initial objects, they have only one unique map to every objects in the category, including themselves. And $Id_{A},Id_{B}$ maps $A,B$ to themselves. Hence we obtain
			\begin{align*}
				& Id_{A}=\psi\circ\varphi \\
				& Id_{B}=\psi\circ\varphi
			.\end{align*}
			So we know $\varphi,\phi$ are isomorphisms.
			\qed
		\item The proof is similar to \emph{(a)}, skipped.
	\end{enumerate}
\end{exercise}

\begin{exercise}
	\leavevmode \vspace{-\baselineskip}
	\begin{enumerate}
		\item In the category of sets, the initial and final object is $\emptyset$ and the class of all sets, respectively.
		\item In the category of rings, the initial and final object is the zero ring and the class of all rings, respectively.
		\item \emph{Skipped (haven't learn topology yet).}
	\end{enumerate}
\end{exercise}

\begin{exercise}
	\leavevmode \vspace{-\baselineskip}
	\begin{itemize}
		\item $(\Leftarrow)$: When $S$ contains no zerodivisors, we prove by contradiction: Assume that under the canonical ring map  $\varphi:A\mapsto S^{-1}A$, there exist $a_{1},a_{2}\in A,\ s.t.\varphi(a_{1})=\varphi(a_{2})$. Now we have $a_{1}/1 = a_{2}/1$, which means
			$$
			\exists s \in S, s.t.\ s(a_{1}-a_{2})=0
			.$$ 
			Which contradicts with the fact that $S$ has no zerodivisors!
		\item $(\Rightarrow)$: When $\varphi:A \mapsto  S^{-1}A$ is injective, we assume $S$ contains a zerodivisors $s$ at which
			$$
			\exists d\in A,\ s.t.\ d\neq 0,\ s\cdot d=0
			.$$ 
			Now let $a_{1},a_{2}\in A,\ s.t.\ a_{1}-a_{2}=d$. Since $\varphi$ is injective,  we must have $\varphi(a_{1})\neq \varphi(a_{2})$. Which contradicts with the fact that
			$$
			s(a_{1}-a_{2})=s\cdot d = 0
			.$$ 
			\qed
	\end{itemize}
\end{exercise}

\begin{exercise}
	Suppose
	\begin{align*}
		& f:A\mapsto B\\
		& \varphi :A\mapsto S^{-1}A
	.\end{align*}
	By the first translation in by author, we have to prove that there an unique ring homomorphism $\psi$ mapping $S^{-1}A$ to $B$. Recall that $\forall a \in A, f(a)=\varphi(a / 1)$. Also, for all $a\in A,s \in S$,
	$$
	\frac{a}{s}\cdot \frac{s}{1}= \frac{a}{1}
	.$$ 
	So
	\begin{align*}
		& \psi( \frac{a}{s} )\cdot \psi( \frac{s}{1} ) = \psi( \frac{a}{1} )\\
		& \psi( \frac{a}{s} )= \psi( \frac{a}{1} )\cdot \psi( \frac{s}{1} )^{-1} \\
	.\end{align*}
	And we can see $\psi( \frac{a}{s} )$ is uniquely defined!
	\qed
\end{exercise}

\begin{exercise}
	We follow the hint from author: Define
	$$
	S^{-1}M= \left\{ \frac{m}{s}\mid m \in M,s \in S \right\} 
	.$$ 
	And $S^{-1}M$ meet the requirement to become a module by: $\forall m_{1},m_{2}\in M,s_{1},s_{2}\in S:$
	$$
	\frac{m_{1}}{s_{1}} = \frac{m_{2}}{s_{2}} \iff \exists s \in S,\ s \left( s_{2}m_{1}-s_{1}m_{2} \right) =0\\
	.$$ 
	$$
	\frac{m_{1}}{s_{1}} + \frac{m_{2}}{s_{2}} = \frac{m_{1}s_{2}+m_{2}s_{1}}{s_{1}s_{2}}
	.$$ 
	And the $S^{-1}A$ module structure
	$$
	\forall a \in A,s_{1},s_{2}\in S,m\in M,\ \frac{a}{s_{1}}\cdot \frac{m}{s_{2}} = \frac{am}{s_{1}s_{2}}
	.$$ 
	Also, we define $M\Rightarrow S^{-1}M$ by
	$$
	m \longrightarrow \frac{m}{1}
	.$$ 
	Now for any $A$-module maps $M\rightarrow N$, we'll prove $\phi$ satisfies the universal proterty, i.e. the map $S^{-1}M\longrightarrow N$, we denote it by $\varphi$, is unique: $\forall \frac{m}{s}\in S^{-1}M$, notice:
	$$
	\begin{cases}
		\alpha(m) \in N \\
		\alpha(s)^{-1} \in N
	\end{cases}
	.$$ 
	Therefore we have $\varphi( \frac{m}{s} ) = \alpha(m) \alpha(s)^{-1}$. So the $\phi$ in the problem satisfy the uniqueness of universal property, so it exist.
	\qed
\end{exercise}

\begin{exercise}
	\leavevmode \vspace{-\baselineskip}
	\begin{enumerate}
		\item Intuitively, we can let
			$$
			\frac{(m_{1},m_{2},\ldots ,m_{n})}{s} \longrightarrow \left( \frac{m_{1}}{s},\ \frac{m_{2}}{s},\ \ldots,\ \frac{m_{n}}{s} \right) 
			.$$ 
			On the other hand, inductively we only have to consider the case $n=2$, and we let
			$$
			\left( \frac{m_{1}}{s_{1}},\ \frac{m_{2}}{s_{2}} \right) \longrightarrow \left( \frac{m_{1}s_{2}}{s_{1}s_{2}},\ \frac{m_{2}s_{1}}{s_{1}s_{2}} \right) 
			.$$ 
			And we're done after induction.
		\item Notice that in the case of direct sum, all but finitely many coordinates is zero. Therefore the proof is exactly the same as the case in \emph{(a)}.
		\item Counterexample: consider group of modules $\{M_{n}\}_{n=1}^{\infty}$, and set of prime numbers $P=\{p_{n}\}_{n=1}^{\infty}$. We consider fraction of $\{M_{n}\}_{n=1}^{\infty}$ by positive integers, we cannot map ${\frac{m_1}{p_1},\frac{m_2}{p_2},...}$ to a proper element in $S^{-1}\left( \bigotimes _{k \in I}M_{k} \right) $ since $p_{1},\ p_{2},\ \ldots$ cannot have a shared denominator.
	\end{enumerate}
\end{exercise}

\begin{exercise}
	More generally, we'll prove that
	$$
	\forall m,\ n \in \mathbb{Z}^{+},\ \mathbb{Z} / (m) \otimes_{\mathbb{Z}}\mathbb{Z} \ (n) \cong \mathbb{Z} / (\mathrm{gcd}(m,n))
	.$$ 
	First, we know $\mathbb{Z} / (m) \otimes_{\mathbb{Z}}\mathbb{Z} \ (n)$ was generated by $1_{\mathbb{Z} / (m)} \otimes 1_{\mathbb{Z} / (n)}$. So means that every elements of $\mathbb{Z} / (m) \otimes_{\mathbb{Z}}\mathbb{Z} \ (n) $ can be written in the form of $k\cdot (1_{\mathbb{Z} / (m)} \otimes 1_{\mathbb{Z} / (n)}),\ k \in \mathbb{Z}^{+}$. Also, $\forall i,\ j \in \mathbb{Z}^{+}$,
	$$
	k\cdot (1_{\mathbb{Z} / (m)} \otimes 1_{\mathbb{Z} / (n)}) = (mi+nj)\cdot k\cdot (1_{\mathbb{Z} / (m)} \otimes 1_{\mathbb{Z} / (n)})
	.$$ 
	And the proof can be done by Euclidean algorithm.
	\qed
\end{exercise}

\begin{exercise}
	\leavevmode \vspace{-\baselineskip}
	\begin{enumerate}
		\item Denote such mapping by $\mathcal{F}$, we'll prove $\mathcal{F}$ is a covariant functor.
			We know $$
			\mathcal{F}:M\longrightarrow \mathcal{F}(M)
			.$$ 
			And for morphism $m:M\longrightarrow m(M)$, we have
			$$
			\mathcal{F}(m):M\otimes N \longrightarrow m(M) \otimes N
			.$$ 
			We need to prove that $\mathcal{F}(\mathrm{id}_{M})=\mathrm{id}_{\mathcal{F}(M)}$ and $ \mathcal{F}(g\circ f)=\mathcal{F}(g)\circ \mathcal{F}(f)$. \\
			First,
			$$
			\mathcal{F}(\mathrm{id}_{M}):M\otimes N \longrightarrow \mathrm{id}_{M}(M)\otimes N = M\otimes N
			.$$ 
			Also,
			$$
			 \mathcal{F}(g\circ f):M\otimes N \longrightarrow g\circ f (M)\otimes N
			.$$ 
			$$
			\mathcal{F}(g)\circ \mathcal{F}(f):M\otimes N \longrightarrow \mathcal{F}(g) \left( f(M)\otimes N \right) =g\circ f (M)\otimes N
			.$$ 
			\qed
		\item \emph{(The proof can be obtianed by checking the mapping property step by step, skipped)}
	\end{enumerate}
\end{exercise}

\begin{exercise}
	\emph{(Trivial.)}\\
	Because from the textbook content right above this question, we see any two maps $T,\ T'$ has a unique $A$-linear map to each other.
\end{exercise}

\begin{exercise}
	Because the construction in \S\emph{1.2.5} is a bilinear map, is unique up to isomorphism.
\end{exercise}

\begin{exercise}
	\leavevmode \vspace{-\baselineskip}
	\begin{enumerate}
		\item 
			\begin{itemize}
				\item $\forall b_{1},\ b_{2}\in B,\ b\otimes_{A}m\in B\otimes M$, we have
					\begin{align*}
						(b_{1}+b_{2})(b\otimes_{A}m)&=(b_{1}+b_{2})b\otimes_{A}m\\
										&=(b_{1}b+b_{2}b)\otimes_{A}m\\
										&=b_{1}b\otimes_{A}m+b_{2}b\otimes_{A}m\\
										&=b(b_{1}\otimes_{A}m)+b(b_{2}\otimes_{A}m)
					.\end{align*}
					\begin{align*}
						(b_{1}b_{2})(b\otimes_{A}m)&=(b_{1}b_{2}b)\otimes_{A}m\\
							     &=b_{1}(b_{2}(b\otimes_{A}m))
					.\end{align*}
					\begin{align*}
						b(b_{1}\otimes_{A}m+b_{2}\otimes_{A}m)&=b(b_{1}\otimes_{A}m)+b(b_{2}\otimes_{A}m)
					.\end{align*}
					% Which means that $B\otimes_{A}M$ meets the requirement to be a $B$-module.
				\item Denote such map by $\mathcal{F}$. We see for ring $A$ and $B\otimes_{A}M\in \mathrm{Mod}_{B}$,
					$$
					 \mathcal{F}: A\longmapsto B\otimes_{A}M
					.$$ 
					And for $\varphi \in \mathrm{Mor}(A),\ \varphi: A\mapsto A'$,
					$$
					\mathcal{F}(\varphi):B\otimes_{A}M\longmapsto B\otimes_{A'}M
					.$$ 
			\end{itemize}
		\item We define addtion as formal sum and multiplication as
			$$
			(b_1\otimes c_1)\cdot (b_2\otimes c_2)=(b_1b_2)\otimes (c_1c_2)
			.$$ 
			And we can easily exam the properties to let it becomes a ring.
	\end{enumerate}
\end{exercise}

\begin{exercise}
	We can define
	$$
	(S^{-1}A)\otimes _{A}M\longrightarrow S^{-1}M
	.$$ 
	As
	$$
	(\frac{a}{s})\otimes m\longmapsto \frac{am}{s}
	.$$ 
	And we can easily check that it meets the requirement to become a isomorphism.
\end{exercise}

\begin{exercise}
	Because $I$ only has finite nonzero elements, so the properties of tensor product can be verified easily.
\end{exercise}

\begin{exercise}
	It's because
	$$
	\alpha\circ\mathrm{pr}_{X} = \beta\circ\mathrm{pr}_{Y}
	.$$ 
\end{exercise}

\begin{exercise}
	\emph{(Skipped temporarily, haven't learnt topology yet.)}
\end{exercise}

\begin{exercise}
	Denote the natural maps
	$$
	\mathrm{pr}_{X}:X\times Y\longrightarrow X
	.$$ 
	$$
	pr_{Y}:X\times Y\longrightarrow Y
	.$$ 
	And the unique map
	$$
	\alpha:X\longrightarrow Z
	.$$ 
	$$
	\beta:Y\longrightarrow Z
	.$$ 
	Now we have to prove
	$$
	\alpha\circ\mathrm{pr}_{X}=\beta\circ\mathrm{pr}_{Y}
	.$$ 
	Which do work since $Z$ is a final diagram, from which we know, up to isomorphisms, it only has a single morphism sending $X\times Y$ to $Z$.
\end{exercise}

\begin{exercise}
	For notations, let
	$$
	\begin{tikzcd}
U \arrow[r, "f_1"] \arrow[d, "g_1"] & V \arrow[d, "h_1"] \\
W \arrow[d, "g_2"] \arrow[r, "f_2"] & X \arrow[d, "h_2"] \\
Y \arrow[r, "f_3"]                  & Z                 
\end{tikzcd}
	.$$ 
	And we can see
	\begin{align*}
		h_2\circ h_1\circ f_1&=h_2\circ f_2\circ g_1\\
				     &=f_3\circ g_2\circ g_1
	.\end{align*}
	\qed
\end{exercise}

\begin{exercise}
	$$
	\begin{tikzcd}
X_1\times_Y X_2 \arrow[rdd] \arrow[rrd] \arrow[rd, "\exists!" description, dotted] &                                       &                                   &   \\
                                                                                   & X_1\times_{Z} X_2 \arrow[d] \arrow[r] & X_2 \arrow[d] \arrow[rdd, dashed] &   \\
                                                                                   & X_1 \arrow[r] \arrow[rrd, dashed]     & Y \arrow[rd]                      &   \\
                                                                                   &                                       &                                   & Z
	\end{tikzcd}
	.$$ 
	Because from the problem we have the commutative diagram
	$$
	\begin{tikzcd}
X_1\times_Y X_2 \arrow[d] \arrow[r] & X_2 \arrow[d, dashed] \\
X_1 \arrow[r, dashed]               & Z                    
	\end{tikzcd}
	.$$ 
	And by the definition of fibered product, there must exist a unqiue factoring map $X_1\times_{Y}X_2\longrightarrow X_{1}\times _{Z}X_2$.
\end{exercise}

\begin{exercise}
	\emph{(Failed, not finished.)}\\
	Given
	$$
	\begin{tikzcd}
X_1\times_Y X_2 \arrow[d] \arrow[r] & X_2 \arrow[d] &   \\
X_1 \arrow[r]                       & Y \arrow[rd]  &   \\
                                    &               & Z
	\end{tikzcd}
	.$$ 
	We need to prove the diagram below is a Cartesian Square:
	$$
	\begin{tikzcd}
X_1\times_Y X_2 \arrow[d] \arrow[r] & X_1\times_Z X_2 \arrow[d] \\
Y \arrow[r]                         & Y\times_Z Y              
	\end{tikzcd}
	.$$ 
	From the previous exercise we know there exist a unique factoring map
	$$
	X_1\times_{Y}X_2\longrightarrow X_{1}\times _{Z}X_2
	.$$ 
	Notice there exist a unique map $X_1\times _{Y}X_2\longrightarrow X_1\times _{Z}X_2$:
	$$
	\begin{tikzcd}
X_1\times_ZX_2 \arrow[d] \arrow[r] \arrow[rd, "\exists!" description, dotted] & X_2 \arrow[rd]                 &                                             \\
X_1 \arrow[rd]                                                                & Y\times_ZY \arrow[d] \arrow[r] & Y \arrow[d] \arrow[ld, Rightarrow, no head] \\
                                                                              & Y \arrow[r]                    & Z                                          
	\end{tikzcd}
	.$$ 
\end{exercise}

\begin{exercise}
	For
	$$
	\begin{tikzcd}
                                                & S                                                       &                                                  \\
S_1 \arrow[r, "\varphi_1"] \arrow[ru, "\psi_1"] & S_1\sqcup S_2 \arrow[u, dotted] & S_2 \arrow[l, "\varphi_2"] \arrow[lu, "\psi_2"']
	\end{tikzcd}
	.$$ 
	For $s_1\in S_1,\ s_2\in S_2$, define
	$$
	\varphi_1(s_1)=(s_1,\cdot ),\ \varphi_2(s_2)=(\cdot ,s_2)
	.$$
	Now for any $\psi_1,\ \psi_2$, we naturally have the induced morphism
	$$
	\Psi :S_1\sqcup S_2\longrightarrow S
	$$ 
	$$
	\Psi:(s_1,\cdot )\longmapsto \psi_1(s_1)
	$$ 
	$$
	\Psi:(\cdot ,s_2)\longmapsto \psi_2(s_2)
	.$$
	Which is unique and make the diagram commutative.
\end{exercise}

\begin{exercise}
	Denote
	$$
	\begin{tikzcd}
D &                                                       &                               \\
  & B\times_AC \arrow[lu, "\varphi" description, dotted] & C \arrow[l] \arrow[llu, "h"'] \\
  & B \arrow[u] \arrow[luu, "g"]                          & A \arrow[l] \arrow[u]        
	\end{tikzcd}
	$$ 
	and we obtain the unique map $\varphi$ by
	$$
	\varphi((a\cdot b)\otimes c)=g(b)
	$$ 
	$$
	\varphi(b\otimes (a\cdot c))=h(c)
	$$ 
\end{exercise}

\begin{exercise}
	$$
	\begin{tikzcd}
A \arrow[d, "\leq1" description, dotted] \arrow[rd, "\leq1" description, dotted] \arrow[rrd] &                       &   \\
X \arrow[r, "\alpha"']                                                                       & Y \arrow[r, "\beta"'] & Z
	\end{tikzcd}
	$$ 
	From the graph we obtian the proof easily.
\end{exercise}

\begin{exercise}
	\emph{(Failed)}
	\begin{enumerate}
		\item 
			\begin{itemize}
				\item When $\pi$ is a monomorphism, we let $X\times _{Y}X=X$ by
					$$
					\begin{tikzcd}
				X \arrow[d, "="'] \arrow[r, "="] & X \arrow[d, "\pi"] \\
				X \arrow[r, "\pi"']              & Y                 
					\end{tikzcd}
					$$ 
					Now for any $A$ satisfying the commutative diagram below, it must factor through $X$ uniquely by the definition of monomorphism.
					$$
					\begin{tikzcd}
				A \arrow[rdd] \arrow[rrd] \arrow[rd, "\exists!" description, dotted] &                                  &                    \\
        		                                                             		& X \arrow[d, "="'] \arrow[r, "="] & X \arrow[d, "\pi"] \\
        		                                                             		& X \arrow[r, "\pi"']              & Y                 
					\end{tikzcd}
					$$ 
				\item When $X\times _{Y}X$ exists, then any $A$ letting the diagram below commute
					$$
					\begin{tikzcd}
A \arrow[rdd] \arrow[rrd] \arrow[rd, "\exists!" description, dotted] &                                &                    \\
                                                                     & X\times_YX \arrow[d] \arrow[r] & X \arrow[d, "\pi"] \\
                                                                     & X \arrow[r, "\pi"']            & Y                 
					\end{tikzcd}
					,$$ 
					whose morphism to $X\times _{Y}X$ will factor through $X\times _{Y}X$ uniquely. Therefore any two map:
					$$
					\alpha,\ \beta:A\longrightarrow X
					$$ 
					can be factorized into
					$$
					\alpha= \alpha'\circ p,\ \beta=\beta'\circ p
					$$ 
					Where $p:A\longrightarrow X\times _{Y}X$ and $\alpha',\ \beta':X\times_{Y}X\longrightarrow X$. \\
					$$
					\begin{tikzcd}
A \arrow[rdd, "\alpha"'] \arrow[rrd, "\beta"] \arrow[rd, "p" description, dotted] &                                                      &                    \\
                                                                                  & X\times_YX \arrow[d, "\alpha'"'] \arrow[r, "\beta'"] & X \arrow[d, "\pi"] \\
                                                                                  & X \arrow[r, "\pi"']                                  & Y                 
					\end{tikzcd}
					$$ 
					So if
					$$
					\pi\circ \alpha=\pi\circ \beta
					$$ 
					We'll have
					$$
					\pi\circ \alpha'\circ p=\pi\circ \beta'\circ p
					$$ 
					Since $X\times _{Y}X$ is a fibered product, $\pi\alpha'=\pi\beta'$.
			\end{itemize}
		\item 
		\end{enumerate}
\end{exercise}

\begin{exercise}
	$$
	\begin{tikzcd}
X_1\times_YX_2 \arrow[rdd] \arrow[rrd] &                                    &               &   \\
                                       & X_1\times_ZX_2 \arrow[d] \arrow[r] & X_2 \arrow[d] &   \\
                                       & X_1 \arrow[r]                      & Y \arrow[rd]  &   \\
                                       &                                    &               & Z
	\end{tikzcd}
	$$ 
	From Exercise \textbf{1.2.R} we know there exist a unique morphism
	$$
	\varphi:X_1\times_{Y}X_2\longrightarrow X_1\times _{Z}X_2
	.$$ 
	Now since $Y\longrightarrow Z$ is monomorphism, from the commutavity of $X_1\times _{Z}X_2\rightarrow X_1\rightarrow Y\rightarrow Z$ and $X_1\times _{Z}X_2\rightarrow X_2\rightarrow Y\rightarrow Z$ we can get the commutavity of
	$$
	\begin{tikzcd}
X_1\times_ZX_2 \arrow[d] \arrow[r] & X_2 \arrow[d] \\
X_1 \arrow[r]                      & Y            
	\end{tikzcd}
	$$ 
	And therefore there's a unique morphism $\psi:X_1\times _{Z}X_2\longrightarrow X_1\times _{Y}X_2$. 
	By their uniqueness, we have
	$$
	\varphi\circ \psi\circ \varphi=\varphi,\ \psi\circ \varphi\circ \psi=\psi
	$$ 
	So $\varphi\circ \psi=\mathrm{id}_{X_1\times _{Z}X_2},\ \psi\circ \varphi=\mathrm{id}_{X_1\times _{Y}X_2}$.
\end{exercise}

\begin{exercise}
	% \leavevmode \vspace{-\baselineskip}
	$$
	\begin{tikzcd}
B \arrow[rrdd] \arrow[dd]                                                    &                                                                               & A' \arrow[lldd, "h"', bend right, shift right=3] \arrow[loop, distance=2em, in=125, out=55] \\
{}                                                                           &                                                                               & {}                                                                                          \\
A \arrow[rruu, "g"', bend right] \arrow[loop, distance=2em, in=305, out=235] & {} \arrow[ru, "i_c"', bend right=67, shift right=2] \arrow[lu, bend right=49] & C \arrow[uu] \arrow[ll, "u" description]                                                   
	\end{tikzcd}
	$$ 
	\begin{enumerate}
		\item We simply let $i_{C}$ correspond to the morphism $A\longrightarrow A'$ and verify the properties stated in the problems. Also, uniqueness can be obtianed by letting $C=A$.
		\item We follow from the hint: Let $C=A'$ and we get a bijection
			$$
			i_{A'}:(\mathrm{Mor}(A',A))\longrightarrow (\mathrm{Mor}(A',A'))
			$$ 
			We pick out $h$ from $\mathrm{Mor(A',A)}$ such that $i_{A'}(h)=\mathrm{id}_{A'}$. From \textbf{(a)} we know $i_{A'}$ can be represented using $g$ :
			$$
			\mathrm{id}_{A'}=i_{A'}(h)=g\circ h
			$$ 
			On the other hand, we prove $i_{A}^{-1}$ can be induced by $h$ using the similar method to \textbf{(a)}.
			$$
			\exists !\ \mathrm{id}_{A}\in \mathrm{Moe(A,A)},\ i_{A}(id_{A})=g\in \mathrm{Mor}(A,A')
			$$ 
			And we can see
			$$
			\mathrm{id}_{A}=i^{-1}(g)=h\circ g
			$$ 
			\qed
	\end{enumerate}
\end{exercise}

\begin{exercise}
	\emph{(Skipped temporarily due to the star.)}
	\begin{enumerate}
		\item We consider the contravariant functors of the morphisms $B\longrightarrow A$.
		\item 
	\end{enumerate}
\end{exercise}

\section{Limits and colimits}
\begin{exercise}
	\emph{(Trivial.)}\\
	Since $\mathscr{I}$ is a partially ordered set, there's at most one morphism for any two object in $\mathscr{I}$. 
\end{exercise}

\begin{exercise}
	$$
	\begin{tikzcd}
X_1\times_ZX_2 \arrow[rd, dotted] \arrow[rdd] \arrow[rrd] &                                    &                                   &   \\
                                                          & X_1\times_YX_2 \arrow[r] \arrow[d] & X_2 \arrow[d] \arrow[rdd, dashed] &   \\
                                                          & X_1 \arrow[r] \arrow[rrd, dashed]  & Y \arrow[rd]                      &   \\
                                                          &                                    &                                   & Z
	\end{tikzcd}
	\begin{tikzcd}
Y\times_{(Y\times_ZY)}(X_1\times_ZX_2) \arrow[d] \arrow[r] & X_1\times_ZX_2 \arrow[d] \\
Y \arrow[r]                                                 & Y\times_ZY              
	\end{tikzcd}
	$$ 
	For $X_1\times _{Y}X_2$:
	\begin{align*}
		X_1\times _{Y}X_2\rightarrow Y&=X_1\times _{Y}X_2\rightarrow X_1\rightarrow Y\\
					      &=X_1\times _{Y}X_2\rightarrow X_2\rightarrow Y
	\end{align*}
	$$
	X_1\times _{Y}X_2\rightarrow Z=X_1\times _{Y}X_2\rightarrow Y\rightarrow Z
	$$ 
	For $Y\times_{(Y\times _{Z}Y)}(X_1\times _{Z}X_2)$:
	\begin{align*}
		Y\times_{(Y\times _{Z}Y)}(X_1\times _{Z}X_2)\rightarrow Y
		&=Y\times_{(Y\times _{Z}Y)}(X_1\times _{Z}X_2)\rightarrow X_1\times _{Z}X_2\rightarrow X_1\rightarrow Y\\
		&=Y\times_{(Y\times _{Z}Y)}(X_1\times _{Z}X_2)\rightarrow X_1\rightarrow Y
	\end{align*}
	$$
	Y\times _{(Y\times _{Z}Y)}(X_1\times _{Z}X_2)\rightarrow Z=
	Y\times _{(Y\times _{Z}Y)}(X_1\times _{Z}X_2)\rightarrow Y\rightarrow Z
	$$ 
	\qed
\end{exercise}

\begin{exercise}
	$$
	\begin{tikzcd}
(a_i)_{i\in \mathscr{I}} \arrow[d] \arrow[rd] &     \\
A_j \arrow[r]                       & A_k
\end{tikzcd}
	$$ 
	Because from the definition of $(a_{i})_{i\in \mathscr{I}}$, we have
	$$
	F(m)(a_{j})=a_{k}
	$$ 
	So $(a_{i})_{i\in \mathscr{I}}A\longrightarrow A_{j}\longrightarrow A_{k}$ is commutative with $(a_{i})_{i\in \mathscr{I}}\longrightarrow A_{k}$.
	\qed
\end{exercise}

\begin{exercise}
	\leavevmode \vspace{-\baselineskip}
	\begin{enumerate}
		\item It means fractions converting into different denominators stands for the same rational number, which is obviously true.
		\item Because we can view the set category as posets, where morphisms in between two objects are unique.
	\end{enumerate}
\end{exercise}

\begin{exercise}
	The proof is done by definition.
\end{exercise}

\begin{exercise}
	\emph{(I suppose it's trivial, skipped.)}
\end{exercise}

\begin{exercise}
	\emph{(The proof is similar to \textbf{1.3.D} I'll skip.)}
\end{exercise}

\begin{exercise}
	
\end{exercise}

\end{document}
