\documentclass{article}
\usepackage{geometry}
\geometry{a4paper,scale=0.8}
\usepackage[utf8]{inputenc}

\usepackage{amsmath, amssymb}
\usepackage{amsthm}
\usepackage{enumitem}
\usepackage{mathrsfs}

\setlist[enumerate,1]{
    label=(\alph*), 
    topsep=0pt, 
    partopsep=0pt, 
    parsep=0pt,
    leftmargin=*
}
\newcommand{\currchapter}{1}
\newtheoremstyle{breakstyle}
	{5pt}
	{10pt}
	{\normalfont}
	{0pt}
	{\bfseries}
	{}
	{\newline}
	{}
\theoremstyle{breakstyle}
\newtheorem{exercise}{Exercise}[section]
\renewcommand{\theexercise}{\currchapter.\thesection.\Alph{exercise}}
\title{Chapter 1 - Just enough category theory to be dangerous}
\author{Flandre}
\date{29 Dec 2025}

\begin{document}
\maketitle

\section{Categories and functors}
\begin{exercise}
	\leavevmode \vspace{-\baselineskip}
		\begin{enumerate}
			\item Since there's only one object, namely $e$, in the category, all the morphisms in such category will send $e$ to $e$.
				\begin{itemize}
					\item The combinations of mrophisms are always associative.
					\item There must exist an identity morphism, which'll be the identity in the group.
					\item Since it's a groupoid, every morphism has a inverse.
				\end{itemize}
			\item We consider a category $\mathscr{C}$ whose objects are $A,B$. Consider
				$$
				\mathrm{Mor}(A,B)=\{id_{A},id_{B}\}
				.$$ 
				and we can see that is's a groupiod without being associative, therefore not a group.
		\end{enumerate}
\end{exercise}

\begin{exercise}
	test
\end{exercise}

\end{document}
