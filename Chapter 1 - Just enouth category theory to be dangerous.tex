\documentclass{article}
\usepackage{geometry}
\geometry{a4paper,scale=0.8}
\usepackage[utf8]{inputenc}

\usepackage{amsmath, amssymb}
\usepackage{amsthm}
\usepackage{enumitem}
\usepackage{mathrsfs}

\setlist[enumerate,1]{
    label=(\alph*), 
    topsep=0pt, 
    partopsep=0pt, 
    parsep=0pt,
    leftmargin=*
}
\newcommand{\currchapter}{1}
\newtheoremstyle{breakstyle}
	{5pt}
	{10pt}
	{\normalfont}
	{0pt}
	{\bfseries}
	{}
	{\newline}
	{}
\theoremstyle{breakstyle}
\newtheorem{exercise}{Exercise}[section]
\renewcommand{\theexercise}{\currchapter.\thesection.\Alph{exercise}}
\title{Chapter 1 - Just enough category theory to be dangerous}
\author{Flandre}
\date{29 Dec 2025}

\begin{document}
\maketitle

\section{Categories and functors}
\begin{exercise}
	\leavevmode \vspace{-\baselineskip}
		\begin{enumerate}
			\item Since there's only one object, namely $e$, in the category, all the morphisms in such category will send $e$ to $e$.
				\begin{itemize}
					\item The combinations of mrophisms are always associative.
					\item There must exist an identity morphism, which'll be the identity in the group.
					\item Since it's a groupoid, every morphism has a inverse.
				\end{itemize}
				\emph{ (In fact every group can be represented using the language of vategory. Consider Cayley's Theorem and we ca see every group element can be regarded as an automorphism, satisfying the case of morphism to a object itself.) }
			\item We consider a category $\mathscr{C}$ whose objects are $A,B$. Consider
				$$
				\mathrm{Mor}(A,B)=\{id_{A},id_{B}\}
				.$$ 
				and we can see that is's a groupiod without being associative, therefore not a group.
		\end{enumerate}
\end{exercise}

\begin{exercise}
	\leavevmode \vspace{-\baselineskip}
		\begin{enumerate}
			\item Similar to the case in \textbf{1.1.A}, we can use exactly the same proof to show that invertible elements of $\mathrm{Mor}(A,A)$ forms a group.
			\item Automorphism groups of objects in \textbf{Example 1.1.2} are those permutation groups permutating the elements in a group.
			\item Automorphism groups of objects in \textbf{Example 1.1.3} are those invertible linear transformations mapping a linear space to itself.
		\end{enumerate}
\end{exercise}

\begin{exercise}
	% \leavevmode \vspace{-\baselineskip}
	% 	\begin{enumerate}
	% 		\item In the case of finite cases, the set of bases of $Vec_{k},Vec_{k}^{\vee}$ is finite. Therefore the functor $(\cdot)^{\vee}$ must be isomorphism.
	% 	\end{enumerate}
	\emph{skipped (require \textbf{1.1.21} to solve.)}
\end{exercise}

\begin{exercise}
	Since it's finite, just find out the bases and we can define a trivial isomorphism between $\mathscr{V}$ and $f.d.Vec_{k}$ by mapping those bases respectively.
\end{exercise}

\section{Universal properties determine an object up to unique isomorphism}

\begin{exercise}
	\leavevmode \vspace{-\baselineskip}
	\begin{enumerate}
		\item For any two initial objects $A,B$, we define
			 $$
			\varphi: A\mapsto B,\psi: B\mapsto A
			.$$ 
			And we can see 
			$$
			\varphi \circ \psi :B\mapsto B,\psi\circ\varphi:A\mapsto A
			.$$ 
			Also, notice since $A,B$ are initial objects, they have only one unique map to every objects in the category, including themselves. And $Id_{A},Id_{B}$ maps $A,B$ to themselves. Hence we obtain
			\begin{align*}
				& Id_{A}=\psi\circ\varphi \\
				& Id_{B}=\psi\circ\varphi
			.\end{align*}
			So we know $\varphi,\phi$ are isomorphisms.
			\qed
		\item The proof is similar to \emph{(a)}, skipped.
	\end{enumerate}
\end{exercise}

\begin{exercise}
	\leavevmode \vspace{-\baselineskip}
	\begin{enumerate}
		\item In the category of sets, the initial and final object is $\emptyset$ and the class of all sets, respectively.
		\item In the category of rings, the initial and final object is the zero ring and the class of all rings, respectively.
		\item \emph{Skipped (haven't learn topology yet).}
	\end{enumerate}
\end{exercise}

\begin{exercise}
	\leavevmode \vspace{-\baselineskip}
	\begin{itemize}
		\item $(\Leftarrow)$: When $S$ contains no zerodivisors, we prove by contradiction: Assume that under the canonical ring map  $\varphi:A\mapsto S^{-1}A$, there exist $a_{1},a_{2}\in A,\ s.t.\varphi(a_{1})=\varphi(a_{2})$. Now we have $a_{1}/1 = a_{2}/1$, which means
			$$
			\exists s \in S, s.t.\ s(a_{1}-a_{2})=0
			.$$ 
			Which contradicts with the fact that $S$ has no zerodivisors!
		\item $(\Rightarrow)$: When $\varphi:A \mapsto  S^{-1}A$ is injective, we assume $S$ contains a zerodivisors $s$ at which
			$$
			\exists d\in A,\ s.t.\ d\neq 0,\ s\cdot d=0
			.$$ 
			Now let $a_{1},a_{2}\in A,\ s.t.\ a_{1}-a_{2}=d$. Since $\varphi$ is injective,  we must have $\varphi(a_{1})\neq \varphi(a_{2})$. Which contradicts with the fact that
			$$
			s(a_{1}-a_{2})=s\cdot d = 0
			.$$ 
			\qed
	\end{itemize}
\end{exercise}

\begin{exercise}
	Suppose
	\begin{align*}
		& f:A\mapsto B\\
		& \varphi :A\mapsto S^{-1}A
	.\end{align*}
	By the first translation in by author, we have to prove that there an unique ring homomorphism $\psi$ mapping $S^{-1}A$ to $B$. Recall that $\forall a \in A, f(a)=\varphi(a / 1)$. Also, for all $a\in A,s \in S$,
	$$
	\frac{a}{s}\cdot \frac{s}{1}= \frac{a}{1}
	.$$ 
	So
	\begin{align*}
		& \psi( \frac{a}{s} )\cdot \psi( \frac{s}{1} ) = \psi( \frac{a}{1} )\\
		& \psi( \frac{a}{s} )= \psi( \frac{a}{1} )\cdot \psi( \frac{s}{1} )^{-1} \\
	.\end{align*}
	And we can see $\psi( \frac{a}{s} )$ is uniquely defined!
	\qed
\end{exercise}

\begin{exercise}
	We follow the hint from author: Define
	$$
	S^{-1}M= \left\{ \frac{m}{s}\mid m \in M,s \in S \right\} 
	.$$ 
	And $S^{-1}M$ meet the requirement to become a module by: $\forall m_{1},m_{2}\in M,s_{1},s_{2}\in S:$
	$$
	\frac{m_{1}}{s_{1}} = \frac{m_{2}}{s_{2}} \iff \exists s \in S,\ s \left( s_{2}m_{1}-s_{1}m_{2} \right) =0\\
	.$$ 
	$$
	\frac{m_{1}}{s_{1}} + \frac{m_{2}}{s_{2}} = \frac{m_{1}s_{2}+m_{2}s_{1}}{s_{1}s_{2}}
	.$$ 
	And the $S^{-1}A$ module structure
	$$
	\forall a \in A,s_{1},s_{2}\in S,m\in M,\ \frac{a}{s_{1}}\cdot \frac{m}{s_{2}} = \frac{am}{s_{1}s_{2}}
	.$$ 
	Also, we define $M\Rightarrow S^{-1}M$ by
	$$
	m \longrightarrow \frac{m}{1}
	.$$ 
	Now for any $A$-module maps $M\rightarrow N$, we'll prove $\phi$ satisfies the universal proterty, i.e. the map $S^{-1}M\longrightarrow N$, we denote it by $\varphi$, is unique: $\forall \frac{m}{s}\in S^{-1}M$, notice:
	$$
	\begin{cases}
		\alpha(m) \in N \\
		\alpha(s)^{-1} \in N
	\end{cases}
	.$$ 
	Therefore we have $\varphi( \frac{m}{s} ) = \alpha(m) \alpha(s)^{-1}$. So the $\phi$ in the problem satisfy the uniqueness of universal property, so it exist.
	\qed
\end{exercise}

\begin{exercise}
	\leavevmode \vspace{-\baselineskip}
	\begin{enumerate}
		\item Intuitively, we can let
			$$
			\frac{(m_{1},m_{2},\ldots ,m_{n})}{s} \longrightarrow \left( \frac{m_{1}}{s},\ \frac{m_{2}}{s},\ \ldots,\ \frac{m_{n}}{s} \right) 
			.$$ 
			On the other hand, inductively we only have to consider the case $n=2$, and we let
			$$
			\left( \frac{m_{1}}{s_{1}},\ \frac{m_{2}}{s_{2}} \right) \longrightarrow \left( \frac{m_{1}s_{2}}{s_{1}s_{2}},\ \frac{m_{2}s_{1}}{s_{1}s_{2}} \right) 
			.$$ 
			And we're done after induction.
		\item Notice that in the case of direct sum, all but finitely many coordinates is zero. Therefore the proof is exactly the same as the case in \emph{(a)}.
		\item Counterexample: consider group of modules $\{M_{n}\}_{n=1}^{\infty}$, and set of prime numbers $P=\{p_{n}\}_{n=1}^{\infty}$. We consider fraction of $\{M_{n}\}_{n=1}^{\infty}$ by positive integers, we cannot map ${\frac{m_1}{p_1},\frac{m_2}{p_2},...}$ to a proper element in $S^{-1}\left( \bigotimes _{k \in I}M_{k} \right) $ since $p_{1},\ p_{2},\ \ldots$ cannot have a shared denominator.
	\end{enumerate}
\end{exercise}

\begin{exercise}
	More generally, we'll prove that
	$$
	\forall m,\ n \in \mathbb{Z}^{+},\ \mathbb{Z} / (m) \otimes_{\mathbb{Z}}\mathbb{Z} \ (n) \cong \mathbb{Z} / (\mathrm{gcd}(m,n))
	.$$ 
	First, we know $\mathbb{Z} / (m) \otimes_{\mathbb{Z}}\mathbb{Z} \ (n)$ was generated by $1_{\mathbb{Z} / (m)} \otimes 1_{\mathbb{Z} / (n)}$. So means that every elements of $\mathbb{Z} / (m) \otimes_{\mathbb{Z}}\mathbb{Z} \ (n) $ can be written in the form of $k\cdot (1_{\mathbb{Z} / (m)} \otimes 1_{\mathbb{Z} / (n)}),\ k \in \mathbb{Z}^{+}$. Also, $\forall i,\ j \in \mathbb{Z}^{+}$,
	$$
	k\cdot (1_{\mathbb{Z} / (m)} \otimes 1_{\mathbb{Z} / (n)}) = (mi+nj)\cdot k\cdot (1_{\mathbb{Z} / (m)} \otimes 1_{\mathbb{Z} / (n)})
	.$$ 
	And the proof can be done by Euclidean algorithm.
	\qed
\end{exercise}

\begin{exercise}
	\leavevmode \vspace{-\baselineskip}
	\begin{enumerate}
		\item Denote such mapping by $\mathcal{F}$, we'll prove $\mathcal{F}$ is a covariant functor.
			We know $$
			\mathcal{F}:M\longrightarrow \mathcal{F}(M)
			.$$ 
			And for morphism $m:M\longrightarrow m(M)$, we have
			$$
			\mathcal{F}(m):M\otimes N \longrightarrow m(M) \otimes N
			.$$ 
			We need to prove that $\mathcal{F}(\mathrm{id}_{M})=\mathrm{id}_{\mathcal{F}(M)}$ and $ \mathcal{F}(g\circ f)=\mathcal{F}(g)\circ \mathcal{F}(f)$. \\
			First,
			$$
			\mathcal{F}(\mathrm{id}_{M}):M\otimes N \longrightarrow \mathrm{id}_{M}(M)\otimes N = M\otimes N
			.$$ 
			Also,
			$$
			 \mathcal{F}(g\circ f):M\otimes N \longrightarrow g\circ f (M)\otimes N
			.$$ 
			$$
			\mathcal{F}(g)\circ \mathcal{F}(f):M\otimes N \longrightarrow \mathcal{F}(g) \left( f(M)\otimes N \right) =g\circ f (M)\otimes N
			.$$ 
			\qed
		\item 
	\end{enumerate}
\end{exercise}



\end{document}
